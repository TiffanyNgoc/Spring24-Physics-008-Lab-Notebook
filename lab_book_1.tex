%%%%%%%%%%%%%%%%%%%%%%%%%%%%%%%%%%%%%%%%%
% Daily Laboratory Book
% LaTeX Template
% Version 1.0 (4/4/12)
%
% This template has been downloaded from:
% http://www.LaTeXTemplates.com
%
% Original author:
% Frank Kuster (http://www.ctan.org/tex-archive/macros/latex/contrib/labbook/)
%
% Important note:
% This template requires the labbook.cls file to be in the same directory as the
% .tex file. The labbook.cls file provides the necessary structure to create the
% lab book.
%
% The \lipsum[#] commands throughout this template generate dummy text
% to fill the template out. These commands should all be removed when 
% writing lab book content.
%
% HOW TO USE THIS TEMPLATE 
% Each day in the lab consists of three main things:
%
% 1. LABDAY: The first thing to put is the \labday{} command with a date in 
% curly brackets, this will make a new page and put the date in big letters 
% at the top.
%
% 2. EXPERIMENT: Next you need to specify what experiment(s) you are 
% working on with an \experiment{} command with the experiment shorthand 
% in the curly brackets. The experiment shorthand is defined in the 
% 'DEFINITION OF EXPERIMENTS' section below, this means you can 
% say \experiment{pcr} and the actual text written to the PDF will be what 
% you set the 'pcr' experiment to be. If the experiment is a one off, you can 
% just write it in the bracket without creating a shorthand. Note: if you don't 
% want to have an experiment, just leave this out and it won't be printed.
%
% 3. CONTENT: Following the experiment is the content, i.e. what progress 
% you made on the experiment that day.
%
%%%%%%%%%%%%%%%%%%%%%%%%%%%%%%%%%%%%%%%%%

%----------------------------------------------------------------------------------------
%	PACKAGES AND OTHER DOCUMENT CONFIGURATIONS
%----------------------------------------------------------------------------------------

\documentclass[idxtotoc,hyperref,openany]{labbook} % 'openany' here removes the gap page between days, erase it to restore this gap; 'oneside' can also be added to remove the shift that odd pages have to the right for easier reading

\usepackage[ 
  backref=page,
  pdfpagelabels=true,
  plainpages=false,
  colorlinks=true,
  bookmarks=true,
  pdfview=FitB]{hyperref} % Required for the hyperlinks within the PDF
  
\usepackage{booktabs} % Required for the top and bottom rules in the table
\usepackage{float} % Required for specifying the exact location of a figure or table
\usepackage{graphicx} % Required for including images2
\usepackage{lipsum} % Used for inserting dummy 'Lorem ipsum' text into the template
\usepackage[shortlabels]{enumitem}

\newcommand{\HRule}{\rule{\linewidth}{0.5mm}} % Command to make the lines in the title page
\setlength\parindent{0pt} % Removes all indentation from paragraphs

%----------------------------------------------------------------------------------------
%	DEFINITION OF EXPERIMENTS
%----------------------------------------------------------------------------------------

\newexperiment{example}{This is an example experiment}
\newexperiment{example2}{This is another example experiment}
\newexperiment{example3}{This is yet another example experiment}
\newexperiment{table}{This shows a sample table}
%\newexperiment{shorthand}{Description of the experiment}

%---------------------------------------------------------------------------------------

\begin{document}

%----------------------------------------------------------------------------------------
%	TITLE PAGE
%----------------------------------------------------------------------------------------

\frontmatter % Use Roman numerals for page numbers
\title{
\begin{center}
\HRule \\[0.4cm]
{\Huge \bfseries Laboratory Journal \\[0.5cm] \Large Bachelor of Science}\\[0.4cm] % Degree
\HRule \\[1.5cm]
\end{center}
}
\author{\Huge Tiffany Pham \\ \\ \LARGE tpham118@ucmerced.edu \\[2cm]} % Your name and email address
\date{Beginning 24 January 2024} % Beginning date
\maketitle

\tableofcontents

\mainmatter % Use Arabic numerals for page numbers

%----------------------------------------------------------------------------------------
%	LAB BOOK CONTENTS
%----------------------------------------------------------------------------------------

% Blank template to use for new days:

%\labday{Day, Date Month Year}

%\experiment{}

%Text

%-----------------------------------------

%\experiment{}

%Text

%----------------------------------------------------------------------------------------

\labday{Wednesday, 24 January 2024}

\experiment{Lab Notebook Prompts}
\begin{enumerate}
    \item Maintaining Your Laboratory Notebook: This course specifies how to maintain and organize information in your lab notebook, which will help you develop this important research skill. Figures 1 – 8 are examples of various lab notebooks from undergraduates, graduate students, professors, and famous scientists in history.
    \begin{enumerate}[(a)] % (a), (b), (c), ...
        \item Which of the examples have characteristics that match the Maintaining Your Laboratory Notebook from the guide? Give the figure number and details of what matches.
    \end{enumerate}
    \item Notebook sections: The examples are just 1-2 pages from long-term research efforts; in practice, scientists develop their own style of keeping a lab notebook. Match each example with the applicable notebook section we’re having you use. Explain your choice.
    \item Reproducibility is important in science. Assuming the examples are typical of how the researcher keeps their lab notebook, which of the examples written in English would be easiest to use for repeating the experiment? Most difficult? Explain.
    \item Explain the value of not deleting or erasing mistakes, errors, or misunderstandings from your lab notebook. Why do we specify that you will need to identify and mark the error, explain how you identified that it was an error, and detail how you corrected it?
    \item Plan for your own notebook.
    \begin{enumerate}[(a)]
        \item What aspects of keeping a lab notebook do you expect to do well throughout the semester? Explain, referencing the rubric.
        \item Some aspects will need more practice. In what areas will TA feedback be most helpful as they grade your lab notebooks?
        \item What questions, if any, do you have about the lab notebook guide and rubric?
    \end{enumerate}
    \item Create a single, multi-page PDF of your answers. If your responses are all on one page, please draw a picture and add a funny comic (keep it safe for work!) on a second page. This is to ensure you know how to create multi-page PDFs. There are instructions above for how to create a PDF if you wrote things up on paper or PDF if you used OneNote without awkwardly cutting off lines or images. \textbf{It is your responsibility to ensure that your PDF is legible and uploaded correctly.}
\end{enumerate}

%-----------------------------------------

\experiment{Lab Notebook Answers} % Multiple experiments can be included in a single day, this allows you to segment what was done each day into separate categories
\begin{enumerate}
    \item Figure 2 is an excellent example of a lab notebook as they include a header and date and even have time slots for what they did during those times. They also included their goals for the week. and their objectives. Plus, their handwriting is neat and legible.
    \item Notebook Sections:
    \begin{enumerate}[(a)]
        \item \textbf{Figure 1:} Fits with the \textbf{handwriting section}; their handwriting is not messy as it is neat and legible.
        \item \textbf{Figure 2:} Fits with the \textbf{page headers section}; they wrote the date, title, and the names of all participants involved in the experiment.
        \item \textbf{Figure 3:} Fits with the \textbf{annotations section}; they used marks on the margins of the paper to bring attention to certain points. They also underlined and drew diagrams to help their notes.
        \item \textbf{Figure 4:} Fits with the \textbf{annotations section}; they underlined the important parts of their notes and also included a model drawing of their experiment.
        \item \textbf{Figure 5:} Fits with the \textbf{date and page numbers section}; they wrote the page number on the top corner of their page.
        \item \textbf{Figure 6:} Fits with the \textbf{title section}; they included the title and objectives of their experiment for that day.
        \item \textbf{Figure 7:} Fits with the \textbf{mistakes section}; they crossed out some of what they wrote but did not erase it. So they showed their mistakes and just continued on after
        \item \textbf{Figure 8:} Fits with the \textbf{mistakes section}; They wrote all their calculations down but didn't erase their mistakes. You can see some calculations that are crossed out.
    \end{enumerate}
    \item Figure 3 would be best to reproduce as they list the steps for what they did clearly, and the handwriting is neat and legible.
    \item It is important not to erase any mistakes, errors, or misunderstandings from your lab notebook as they are fundamental to the scientific process. In scientific research, identifying and correcting errors in a lab notebook are integral processes to uphold transparency, reliability, and the advancement of knowledge. Usually, identifying errors will involve meticulously reviewing experimental data, cross-checking with established protocols, collaborative discussions, and comparison with controls. The documentation of error identification includes clear annotations, contextual information, and an analysis of the circumstances surrounding the mistake. Detailing the correction process requires a step-by-step account of the actions taken, presentation of corrected data alongside original records, and a reflection on lessons learned. By doing this, it emphasizes collaboration, acknowledging team contributions, and discussing preventive measures that will enhance the overall integrity of the scientific process. This comprehensive approach ensures accurate and reproducible results and fosters a culture of continuous improvement and accountability within the scientific community.
    \item Planning Notebook:
    \begin{enumerate}[(a)]
        \item Based on what is written on the rubric, I think I could keep up my general notebook maintenance, such as writing down the date, page numbers, page headers, lab titles, etc. Usually, the upkeep is pretty easy, and because my lab notebook is online, it is easy to adjust and edit the format. I think I would also do well on the objectives, purpose, and prediction criteria. Usually, the objectives, purpose, and prediction criteria are some of the first things you write down, and you usually copy and paste these down, so these are pretty straightforward and easy. Another part of the rubric I think I could do well on is the experimental plan/ procedure, as some are also already known, and some are copied, pasted, and straightforward. Then, anything else below those criteria, I could do, but I probably won't be able to complete them 100\% or perfectly.
        \item In my opinion, I think I would get the most out when the feedback I get is on my data; data analysis and interpretations and also my lab reflections.
        \item I have no questions.
    \end{enumerate}
\end{enumerate}

%----------------------------------------------------------------------------------------

\labday{Thursday, 25 January 2024}

\experiment{Lab 02 (Prelab) - Is Your Phone a Good Protractor?}






\experiment{table}

\begin{table}[H]
\begin{tabular}{l l l}
\toprule
\textbf{Groups} & \textbf{Treatment X} & \textbf{Treatment Y} \\
\toprule
1 & 0.2 & 0.8\\
2 & 0.17 & 0.7\\
3 & 0.24 & 0.75\\
4 & 0.68 & 0.3\\
\bottomrule
\end{tabular}
\caption{The effects of treatments X and Y on the four groups studied.}
\label{tab:treatments_xy}
\end{table}

Table \ref{tab:treatments_xy} shows that groups 1-3 reacted similarly to the two treatments but group 4 showed a reversed reaction.

%----------------------------------------------------------------------------------------

\labday{Saturday, Date Month Year}

\experiment{Bulleted list example} % You don't need to make a \newexperiment if you only plan on referencing it once

This is a bulleted list:

\begin{itemize}
\item Item 1
\item Item 2
\item \ldots and so on
\end{itemize}

%-----------------------------------------

\experiment{example}

\lipsum[6]

%-----------------------------------------

\experiment{example2}

\lipsum[7]

%----------------------------------------------------------------------------------------
%	FORMULAE AND MEDIA RECIPES
%----------------------------------------------------------------------------------------

\labday{} % We don't want a date here so we make the labday blank

\begin{center}
\HRule \\[0.4cm]
{\huge \textbf{Formulae and Media Recipes}}\\[0.4cm] % Heading
\HRule \\[1.5cm]
\end{center}

%----------------------------------------------------------------------------------------
%	MEDIA RECIPES
%----------------------------------------------------------------------------------------

\newpage

\huge \textbf{Media} \\ \\

\normalsize \textbf{Media 1}\\
\begin{table}[H]
\begin{tabular}{l l l}
\toprule
\textbf{Compound} & \textbf{1L} & \textbf{0.5L}\\
\toprule
Compound 1 & 10g & 5g\\
Compound 2 & 20g & 10g\\
\bottomrule
\end{tabular}
\caption{Ingredients in Media 1.}
\label{tab:med1}
\end{table}

%-----------------------------------------

%\textbf{Media 2}\\ \\

%Description

%----------------------------------------------------------------------------------------
%	FORMULAE
%----------------------------------------------------------------------------------------

\newpage

\huge \textbf{Formulae} \\ \\

\normalsize \textbf{Formula 1 - Pythagorean theorem}\\ \\
$a^2 + b^2 = c^2$\\ \\

%-----------------------------------------

%\textbf{Formula X - Description}\\ \\

%Formula

%----------------------------------------------------------------------------------------

\end{document}